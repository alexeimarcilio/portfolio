\documentclass[11pt]{article}

    \usepackage[breakable]{tcolorbox}
    \usepackage{parskip} % Stop auto-indenting (to mimic markdown behaviour)
    
    \usepackage{iftex}
    \ifPDFTeX
    	\usepackage[T1]{fontenc}
    	\usepackage{mathpazo}
    \else
    	\usepackage{fontspec}
    \fi

    % Basic figure setup, for now with no caption control since it's done
    % automatically by Pandoc (which extracts ![](path) syntax from Markdown).
    \usepackage{graphicx}
    % Maintain compatibility with old templates. Remove in nbconvert 6.0
    \let\Oldincludegraphics\includegraphics
    % Ensure that by default, figures have no caption (until we provide a
    % proper Figure object with a Caption API and a way to capture that
    % in the conversion process - todo).
    \usepackage{caption}
    \DeclareCaptionFormat{nocaption}{}
    \captionsetup{format=nocaption,aboveskip=0pt,belowskip=0pt}

    \usepackage[Export]{adjustbox} % Used to constrain images to a maximum size
    \adjustboxset{max size={0.9\linewidth}{0.9\paperheight}}
    \usepackage{float}
    \floatplacement{figure}{H} % forces figures to be placed at the correct location
    \usepackage{xcolor} % Allow colors to be defined
    \usepackage{enumerate} % Needed for markdown enumerations to work
    \usepackage{geometry} % Used to adjust the document margins
    \usepackage{amsmath} % Equations
    \usepackage{amssymb} % Equations
    \usepackage{textcomp} % defines textquotesingle
    % Hack from http://tex.stackexchange.com/a/47451/13684:
    \AtBeginDocument{%
        \def\PYZsq{\textquotesingle}% Upright quotes in Pygmentized code
    }
    \usepackage{upquote} % Upright quotes for verbatim code
    \usepackage{eurosym} % defines \euro
    \usepackage[mathletters]{ucs} % Extended unicode (utf-8) support
    \usepackage{fancyvrb} % verbatim replacement that allows latex
    \usepackage{grffile} % extends the file name processing of package graphics 
                         % to support a larger range
    \makeatletter % fix for grffile with XeLaTeX
    \def\Gread@@xetex#1{%
      \IfFileExists{"\Gin@base".bb}%
      {\Gread@eps{\Gin@base.bb}}%
      {\Gread@@xetex@aux#1}%
    }
    \makeatother

    % The hyperref package gives us a pdf with properly built
    % internal navigation ('pdf bookmarks' for the table of contents,
    % internal cross-reference links, web links for URLs, etc.)
    \usepackage{hyperref}
    % The default LaTeX title has an obnoxious amount of whitespace. By default,
    % titling removes some of it. It also provides customization options.
    \usepackage{titling}
    \usepackage{longtable} % longtable support required by pandoc >1.10
    \usepackage{booktabs}  % table support for pandoc > 1.12.2
    \usepackage[inline]{enumitem} % IRkernel/repr support (it uses the enumerate* environment)
    \usepackage[normalem]{ulem} % ulem is needed to support strikethroughs (\sout)
                                % normalem makes italics be italics, not underlines
    \usepackage{mathrsfs}
    

    
    % Colors for the hyperref package
    \definecolor{urlcolor}{rgb}{0,.145,.698}
    \definecolor{linkcolor}{rgb}{.71,0.21,0.01}
    \definecolor{citecolor}{rgb}{.12,.54,.11}

    % ANSI colors
    \definecolor{ansi-black}{HTML}{3E424D}
    \definecolor{ansi-black-intense}{HTML}{282C36}
    \definecolor{ansi-red}{HTML}{E75C58}
    \definecolor{ansi-red-intense}{HTML}{B22B31}
    \definecolor{ansi-green}{HTML}{00A250}
    \definecolor{ansi-green-intense}{HTML}{007427}
    \definecolor{ansi-yellow}{HTML}{DDB62B}
    \definecolor{ansi-yellow-intense}{HTML}{B27D12}
    \definecolor{ansi-blue}{HTML}{208FFB}
    \definecolor{ansi-blue-intense}{HTML}{0065CA}
    \definecolor{ansi-magenta}{HTML}{D160C4}
    \definecolor{ansi-magenta-intense}{HTML}{A03196}
    \definecolor{ansi-cyan}{HTML}{60C6C8}
    \definecolor{ansi-cyan-intense}{HTML}{258F8F}
    \definecolor{ansi-white}{HTML}{C5C1B4}
    \definecolor{ansi-white-intense}{HTML}{A1A6B2}
    \definecolor{ansi-default-inverse-fg}{HTML}{FFFFFF}
    \definecolor{ansi-default-inverse-bg}{HTML}{000000}

    % commands and environments needed by pandoc snippets
    % extracted from the output of `pandoc -s`
    \providecommand{\tightlist}{%
      \setlength{\itemsep}{0pt}\setlength{\parskip}{0pt}}
    \DefineVerbatimEnvironment{Highlighting}{Verbatim}{commandchars=\\\{\}}
    % Add ',fontsize=\small' for more characters per line
    \newenvironment{Shaded}{}{}
    \newcommand{\KeywordTok}[1]{\textcolor[rgb]{0.00,0.44,0.13}{\textbf{{#1}}}}
    \newcommand{\DataTypeTok}[1]{\textcolor[rgb]{0.56,0.13,0.00}{{#1}}}
    \newcommand{\DecValTok}[1]{\textcolor[rgb]{0.25,0.63,0.44}{{#1}}}
    \newcommand{\BaseNTok}[1]{\textcolor[rgb]{0.25,0.63,0.44}{{#1}}}
    \newcommand{\FloatTok}[1]{\textcolor[rgb]{0.25,0.63,0.44}{{#1}}}
    \newcommand{\CharTok}[1]{\textcolor[rgb]{0.25,0.44,0.63}{{#1}}}
    \newcommand{\StringTok}[1]{\textcolor[rgb]{0.25,0.44,0.63}{{#1}}}
    \newcommand{\CommentTok}[1]{\textcolor[rgb]{0.38,0.63,0.69}{\textit{{#1}}}}
    \newcommand{\OtherTok}[1]{\textcolor[rgb]{0.00,0.44,0.13}{{#1}}}
    \newcommand{\AlertTok}[1]{\textcolor[rgb]{1.00,0.00,0.00}{\textbf{{#1}}}}
    \newcommand{\FunctionTok}[1]{\textcolor[rgb]{0.02,0.16,0.49}{{#1}}}
    \newcommand{\RegionMarkerTok}[1]{{#1}}
    \newcommand{\ErrorTok}[1]{\textcolor[rgb]{1.00,0.00,0.00}{\textbf{{#1}}}}
    \newcommand{\NormalTok}[1]{{#1}}
    
    % Additional commands for more recent versions of Pandoc
    \newcommand{\ConstantTok}[1]{\textcolor[rgb]{0.53,0.00,0.00}{{#1}}}
    \newcommand{\SpecialCharTok}[1]{\textcolor[rgb]{0.25,0.44,0.63}{{#1}}}
    \newcommand{\VerbatimStringTok}[1]{\textcolor[rgb]{0.25,0.44,0.63}{{#1}}}
    \newcommand{\SpecialStringTok}[1]{\textcolor[rgb]{0.73,0.40,0.53}{{#1}}}
    \newcommand{\ImportTok}[1]{{#1}}
    \newcommand{\DocumentationTok}[1]{\textcolor[rgb]{0.73,0.13,0.13}{\textit{{#1}}}}
    \newcommand{\AnnotationTok}[1]{\textcolor[rgb]{0.38,0.63,0.69}{\textbf{\textit{{#1}}}}}
    \newcommand{\CommentVarTok}[1]{\textcolor[rgb]{0.38,0.63,0.69}{\textbf{\textit{{#1}}}}}
    \newcommand{\VariableTok}[1]{\textcolor[rgb]{0.10,0.09,0.49}{{#1}}}
    \newcommand{\ControlFlowTok}[1]{\textcolor[rgb]{0.00,0.44,0.13}{\textbf{{#1}}}}
    \newcommand{\OperatorTok}[1]{\textcolor[rgb]{0.40,0.40,0.40}{{#1}}}
    \newcommand{\BuiltInTok}[1]{{#1}}
    \newcommand{\ExtensionTok}[1]{{#1}}
    \newcommand{\PreprocessorTok}[1]{\textcolor[rgb]{0.74,0.48,0.00}{{#1}}}
    \newcommand{\AttributeTok}[1]{\textcolor[rgb]{0.49,0.56,0.16}{{#1}}}
    \newcommand{\InformationTok}[1]{\textcolor[rgb]{0.38,0.63,0.69}{\textbf{\textit{{#1}}}}}
    \newcommand{\WarningTok}[1]{\textcolor[rgb]{0.38,0.63,0.69}{\textbf{\textit{{#1}}}}}
    
    
    % Define a nice break command that doesn't care if a line doesn't already
    % exist.
    \def\br{\hspace*{\fill} \\* }
    % Math Jax compatibility definitions
    \def\gt{>}
    \def\lt{<}
    \let\Oldtex\TeX
    \let\Oldlatex\LaTeX
    \renewcommand{\TeX}{\textrm{\Oldtex}}
    \renewcommand{\LaTeX}{\textrm{\Oldlatex}}
    % Document parameters
    % Document title
    \title{MktOrMonth}
    
    
    
    
    
% Pygments definitions
\makeatletter
\def\PY@reset{\let\PY@it=\relax \let\PY@bf=\relax%
    \let\PY@ul=\relax \let\PY@tc=\relax%
    \let\PY@bc=\relax \let\PY@ff=\relax}
\def\PY@tok#1{\csname PY@tok@#1\endcsname}
\def\PY@toks#1+{\ifx\relax#1\empty\else%
    \PY@tok{#1}\expandafter\PY@toks\fi}
\def\PY@do#1{\PY@bc{\PY@tc{\PY@ul{%
    \PY@it{\PY@bf{\PY@ff{#1}}}}}}}
\def\PY#1#2{\PY@reset\PY@toks#1+\relax+\PY@do{#2}}

\expandafter\def\csname PY@tok@w\endcsname{\def\PY@tc##1{\textcolor[rgb]{0.73,0.73,0.73}{##1}}}
\expandafter\def\csname PY@tok@c\endcsname{\let\PY@it=\textit\def\PY@tc##1{\textcolor[rgb]{0.25,0.50,0.50}{##1}}}
\expandafter\def\csname PY@tok@cp\endcsname{\def\PY@tc##1{\textcolor[rgb]{0.74,0.48,0.00}{##1}}}
\expandafter\def\csname PY@tok@k\endcsname{\let\PY@bf=\textbf\def\PY@tc##1{\textcolor[rgb]{0.00,0.50,0.00}{##1}}}
\expandafter\def\csname PY@tok@kp\endcsname{\def\PY@tc##1{\textcolor[rgb]{0.00,0.50,0.00}{##1}}}
\expandafter\def\csname PY@tok@kt\endcsname{\def\PY@tc##1{\textcolor[rgb]{0.69,0.00,0.25}{##1}}}
\expandafter\def\csname PY@tok@o\endcsname{\def\PY@tc##1{\textcolor[rgb]{0.40,0.40,0.40}{##1}}}
\expandafter\def\csname PY@tok@ow\endcsname{\let\PY@bf=\textbf\def\PY@tc##1{\textcolor[rgb]{0.67,0.13,1.00}{##1}}}
\expandafter\def\csname PY@tok@nb\endcsname{\def\PY@tc##1{\textcolor[rgb]{0.00,0.50,0.00}{##1}}}
\expandafter\def\csname PY@tok@nf\endcsname{\def\PY@tc##1{\textcolor[rgb]{0.00,0.00,1.00}{##1}}}
\expandafter\def\csname PY@tok@nc\endcsname{\let\PY@bf=\textbf\def\PY@tc##1{\textcolor[rgb]{0.00,0.00,1.00}{##1}}}
\expandafter\def\csname PY@tok@nn\endcsname{\let\PY@bf=\textbf\def\PY@tc##1{\textcolor[rgb]{0.00,0.00,1.00}{##1}}}
\expandafter\def\csname PY@tok@ne\endcsname{\let\PY@bf=\textbf\def\PY@tc##1{\textcolor[rgb]{0.82,0.25,0.23}{##1}}}
\expandafter\def\csname PY@tok@nv\endcsname{\def\PY@tc##1{\textcolor[rgb]{0.10,0.09,0.49}{##1}}}
\expandafter\def\csname PY@tok@no\endcsname{\def\PY@tc##1{\textcolor[rgb]{0.53,0.00,0.00}{##1}}}
\expandafter\def\csname PY@tok@nl\endcsname{\def\PY@tc##1{\textcolor[rgb]{0.63,0.63,0.00}{##1}}}
\expandafter\def\csname PY@tok@ni\endcsname{\let\PY@bf=\textbf\def\PY@tc##1{\textcolor[rgb]{0.60,0.60,0.60}{##1}}}
\expandafter\def\csname PY@tok@na\endcsname{\def\PY@tc##1{\textcolor[rgb]{0.49,0.56,0.16}{##1}}}
\expandafter\def\csname PY@tok@nt\endcsname{\let\PY@bf=\textbf\def\PY@tc##1{\textcolor[rgb]{0.00,0.50,0.00}{##1}}}
\expandafter\def\csname PY@tok@nd\endcsname{\def\PY@tc##1{\textcolor[rgb]{0.67,0.13,1.00}{##1}}}
\expandafter\def\csname PY@tok@s\endcsname{\def\PY@tc##1{\textcolor[rgb]{0.73,0.13,0.13}{##1}}}
\expandafter\def\csname PY@tok@sd\endcsname{\let\PY@it=\textit\def\PY@tc##1{\textcolor[rgb]{0.73,0.13,0.13}{##1}}}
\expandafter\def\csname PY@tok@si\endcsname{\let\PY@bf=\textbf\def\PY@tc##1{\textcolor[rgb]{0.73,0.40,0.53}{##1}}}
\expandafter\def\csname PY@tok@se\endcsname{\let\PY@bf=\textbf\def\PY@tc##1{\textcolor[rgb]{0.73,0.40,0.13}{##1}}}
\expandafter\def\csname PY@tok@sr\endcsname{\def\PY@tc##1{\textcolor[rgb]{0.73,0.40,0.53}{##1}}}
\expandafter\def\csname PY@tok@ss\endcsname{\def\PY@tc##1{\textcolor[rgb]{0.10,0.09,0.49}{##1}}}
\expandafter\def\csname PY@tok@sx\endcsname{\def\PY@tc##1{\textcolor[rgb]{0.00,0.50,0.00}{##1}}}
\expandafter\def\csname PY@tok@m\endcsname{\def\PY@tc##1{\textcolor[rgb]{0.40,0.40,0.40}{##1}}}
\expandafter\def\csname PY@tok@gh\endcsname{\let\PY@bf=\textbf\def\PY@tc##1{\textcolor[rgb]{0.00,0.00,0.50}{##1}}}
\expandafter\def\csname PY@tok@gu\endcsname{\let\PY@bf=\textbf\def\PY@tc##1{\textcolor[rgb]{0.50,0.00,0.50}{##1}}}
\expandafter\def\csname PY@tok@gd\endcsname{\def\PY@tc##1{\textcolor[rgb]{0.63,0.00,0.00}{##1}}}
\expandafter\def\csname PY@tok@gi\endcsname{\def\PY@tc##1{\textcolor[rgb]{0.00,0.63,0.00}{##1}}}
\expandafter\def\csname PY@tok@gr\endcsname{\def\PY@tc##1{\textcolor[rgb]{1.00,0.00,0.00}{##1}}}
\expandafter\def\csname PY@tok@ge\endcsname{\let\PY@it=\textit}
\expandafter\def\csname PY@tok@gs\endcsname{\let\PY@bf=\textbf}
\expandafter\def\csname PY@tok@gp\endcsname{\let\PY@bf=\textbf\def\PY@tc##1{\textcolor[rgb]{0.00,0.00,0.50}{##1}}}
\expandafter\def\csname PY@tok@go\endcsname{\def\PY@tc##1{\textcolor[rgb]{0.53,0.53,0.53}{##1}}}
\expandafter\def\csname PY@tok@gt\endcsname{\def\PY@tc##1{\textcolor[rgb]{0.00,0.27,0.87}{##1}}}
\expandafter\def\csname PY@tok@err\endcsname{\def\PY@bc##1{\setlength{\fboxsep}{0pt}\fcolorbox[rgb]{1.00,0.00,0.00}{1,1,1}{\strut ##1}}}
\expandafter\def\csname PY@tok@kc\endcsname{\let\PY@bf=\textbf\def\PY@tc##1{\textcolor[rgb]{0.00,0.50,0.00}{##1}}}
\expandafter\def\csname PY@tok@kd\endcsname{\let\PY@bf=\textbf\def\PY@tc##1{\textcolor[rgb]{0.00,0.50,0.00}{##1}}}
\expandafter\def\csname PY@tok@kn\endcsname{\let\PY@bf=\textbf\def\PY@tc##1{\textcolor[rgb]{0.00,0.50,0.00}{##1}}}
\expandafter\def\csname PY@tok@kr\endcsname{\let\PY@bf=\textbf\def\PY@tc##1{\textcolor[rgb]{0.00,0.50,0.00}{##1}}}
\expandafter\def\csname PY@tok@bp\endcsname{\def\PY@tc##1{\textcolor[rgb]{0.00,0.50,0.00}{##1}}}
\expandafter\def\csname PY@tok@fm\endcsname{\def\PY@tc##1{\textcolor[rgb]{0.00,0.00,1.00}{##1}}}
\expandafter\def\csname PY@tok@vc\endcsname{\def\PY@tc##1{\textcolor[rgb]{0.10,0.09,0.49}{##1}}}
\expandafter\def\csname PY@tok@vg\endcsname{\def\PY@tc##1{\textcolor[rgb]{0.10,0.09,0.49}{##1}}}
\expandafter\def\csname PY@tok@vi\endcsname{\def\PY@tc##1{\textcolor[rgb]{0.10,0.09,0.49}{##1}}}
\expandafter\def\csname PY@tok@vm\endcsname{\def\PY@tc##1{\textcolor[rgb]{0.10,0.09,0.49}{##1}}}
\expandafter\def\csname PY@tok@sa\endcsname{\def\PY@tc##1{\textcolor[rgb]{0.73,0.13,0.13}{##1}}}
\expandafter\def\csname PY@tok@sb\endcsname{\def\PY@tc##1{\textcolor[rgb]{0.73,0.13,0.13}{##1}}}
\expandafter\def\csname PY@tok@sc\endcsname{\def\PY@tc##1{\textcolor[rgb]{0.73,0.13,0.13}{##1}}}
\expandafter\def\csname PY@tok@dl\endcsname{\def\PY@tc##1{\textcolor[rgb]{0.73,0.13,0.13}{##1}}}
\expandafter\def\csname PY@tok@s2\endcsname{\def\PY@tc##1{\textcolor[rgb]{0.73,0.13,0.13}{##1}}}
\expandafter\def\csname PY@tok@sh\endcsname{\def\PY@tc##1{\textcolor[rgb]{0.73,0.13,0.13}{##1}}}
\expandafter\def\csname PY@tok@s1\endcsname{\def\PY@tc##1{\textcolor[rgb]{0.73,0.13,0.13}{##1}}}
\expandafter\def\csname PY@tok@mb\endcsname{\def\PY@tc##1{\textcolor[rgb]{0.40,0.40,0.40}{##1}}}
\expandafter\def\csname PY@tok@mf\endcsname{\def\PY@tc##1{\textcolor[rgb]{0.40,0.40,0.40}{##1}}}
\expandafter\def\csname PY@tok@mh\endcsname{\def\PY@tc##1{\textcolor[rgb]{0.40,0.40,0.40}{##1}}}
\expandafter\def\csname PY@tok@mi\endcsname{\def\PY@tc##1{\textcolor[rgb]{0.40,0.40,0.40}{##1}}}
\expandafter\def\csname PY@tok@il\endcsname{\def\PY@tc##1{\textcolor[rgb]{0.40,0.40,0.40}{##1}}}
\expandafter\def\csname PY@tok@mo\endcsname{\def\PY@tc##1{\textcolor[rgb]{0.40,0.40,0.40}{##1}}}
\expandafter\def\csname PY@tok@ch\endcsname{\let\PY@it=\textit\def\PY@tc##1{\textcolor[rgb]{0.25,0.50,0.50}{##1}}}
\expandafter\def\csname PY@tok@cm\endcsname{\let\PY@it=\textit\def\PY@tc##1{\textcolor[rgb]{0.25,0.50,0.50}{##1}}}
\expandafter\def\csname PY@tok@cpf\endcsname{\let\PY@it=\textit\def\PY@tc##1{\textcolor[rgb]{0.25,0.50,0.50}{##1}}}
\expandafter\def\csname PY@tok@c1\endcsname{\let\PY@it=\textit\def\PY@tc##1{\textcolor[rgb]{0.25,0.50,0.50}{##1}}}
\expandafter\def\csname PY@tok@cs\endcsname{\let\PY@it=\textit\def\PY@tc##1{\textcolor[rgb]{0.25,0.50,0.50}{##1}}}

\def\PYZbs{\char`\\}
\def\PYZus{\char`\_}
\def\PYZob{\char`\{}
\def\PYZcb{\char`\}}
\def\PYZca{\char`\^}
\def\PYZam{\char`\&}
\def\PYZlt{\char`\<}
\def\PYZgt{\char`\>}
\def\PYZsh{\char`\#}
\def\PYZpc{\char`\%}
\def\PYZdl{\char`\$}
\def\PYZhy{\char`\-}
\def\PYZsq{\char`\'}
\def\PYZdq{\char`\"}
\def\PYZti{\char`\~}
% for compatibility with earlier versions
\def\PYZat{@}
\def\PYZlb{[}
\def\PYZrb{]}
\makeatother


    % For linebreaks inside Verbatim environment from package fancyvrb. 
    \makeatletter
        \newbox\Wrappedcontinuationbox 
        \newbox\Wrappedvisiblespacebox 
        \newcommand*\Wrappedvisiblespace {\textcolor{red}{\textvisiblespace}} 
        \newcommand*\Wrappedcontinuationsymbol {\textcolor{red}{\llap{\tiny$\m@th\hookrightarrow$}}} 
        \newcommand*\Wrappedcontinuationindent {3ex } 
        \newcommand*\Wrappedafterbreak {\kern\Wrappedcontinuationindent\copy\Wrappedcontinuationbox} 
        % Take advantage of the already applied Pygments mark-up to insert 
        % potential linebreaks for TeX processing. 
        %        {, <, #, %, $, ' and ": go to next line. 
        %        _, }, ^, &, >, - and ~: stay at end of broken line. 
        % Use of \textquotesingle for straight quote. 
        \newcommand*\Wrappedbreaksatspecials {% 
            \def\PYGZus{\discretionary{\char`\_}{\Wrappedafterbreak}{\char`\_}}% 
            \def\PYGZob{\discretionary{}{\Wrappedafterbreak\char`\{}{\char`\{}}% 
            \def\PYGZcb{\discretionary{\char`\}}{\Wrappedafterbreak}{\char`\}}}% 
            \def\PYGZca{\discretionary{\char`\^}{\Wrappedafterbreak}{\char`\^}}% 
            \def\PYGZam{\discretionary{\char`\&}{\Wrappedafterbreak}{\char`\&}}% 
            \def\PYGZlt{\discretionary{}{\Wrappedafterbreak\char`\<}{\char`\<}}% 
            \def\PYGZgt{\discretionary{\char`\>}{\Wrappedafterbreak}{\char`\>}}% 
            \def\PYGZsh{\discretionary{}{\Wrappedafterbreak\char`\#}{\char`\#}}% 
            \def\PYGZpc{\discretionary{}{\Wrappedafterbreak\char`\%}{\char`\%}}% 
            \def\PYGZdl{\discretionary{}{\Wrappedafterbreak\char`\$}{\char`\$}}% 
            \def\PYGZhy{\discretionary{\char`\-}{\Wrappedafterbreak}{\char`\-}}% 
            \def\PYGZsq{\discretionary{}{\Wrappedafterbreak\textquotesingle}{\textquotesingle}}% 
            \def\PYGZdq{\discretionary{}{\Wrappedafterbreak\char`\"}{\char`\"}}% 
            \def\PYGZti{\discretionary{\char`\~}{\Wrappedafterbreak}{\char`\~}}% 
        } 
        % Some characters . , ; ? ! / are not pygmentized. 
        % This macro makes them "active" and they will insert potential linebreaks 
        \newcommand*\Wrappedbreaksatpunct {% 
            \lccode`\~`\.\lowercase{\def~}{\discretionary{\hbox{\char`\.}}{\Wrappedafterbreak}{\hbox{\char`\.}}}% 
            \lccode`\~`\,\lowercase{\def~}{\discretionary{\hbox{\char`\,}}{\Wrappedafterbreak}{\hbox{\char`\,}}}% 
            \lccode`\~`\;\lowercase{\def~}{\discretionary{\hbox{\char`\;}}{\Wrappedafterbreak}{\hbox{\char`\;}}}% 
            \lccode`\~`\:\lowercase{\def~}{\discretionary{\hbox{\char`\:}}{\Wrappedafterbreak}{\hbox{\char`\:}}}% 
            \lccode`\~`\?\lowercase{\def~}{\discretionary{\hbox{\char`\?}}{\Wrappedafterbreak}{\hbox{\char`\?}}}% 
            \lccode`\~`\!\lowercase{\def~}{\discretionary{\hbox{\char`\!}}{\Wrappedafterbreak}{\hbox{\char`\!}}}% 
            \lccode`\~`\/\lowercase{\def~}{\discretionary{\hbox{\char`\/}}{\Wrappedafterbreak}{\hbox{\char`\/}}}% 
            \catcode`\.\active
            \catcode`\,\active 
            \catcode`\;\active
            \catcode`\:\active
            \catcode`\?\active
            \catcode`\!\active
            \catcode`\/\active 
            \lccode`\~`\~ 	
        }
    \makeatother

    \let\OriginalVerbatim=\Verbatim
    \makeatletter
    \renewcommand{\Verbatim}[1][1]{%
        %\parskip\z@skip
        \sbox\Wrappedcontinuationbox {\Wrappedcontinuationsymbol}%
        \sbox\Wrappedvisiblespacebox {\FV@SetupFont\Wrappedvisiblespace}%
        \def\FancyVerbFormatLine ##1{\hsize\linewidth
            \vtop{\raggedright\hyphenpenalty\z@\exhyphenpenalty\z@
                \doublehyphendemerits\z@\finalhyphendemerits\z@
                \strut ##1\strut}%
        }%
        % If the linebreak is at a space, the latter will be displayed as visible
        % space at end of first line, and a continuation symbol starts next line.
        % Stretch/shrink are however usually zero for typewriter font.
        \def\FV@Space {%
            \nobreak\hskip\z@ plus\fontdimen3\font minus\fontdimen4\font
            \discretionary{\copy\Wrappedvisiblespacebox}{\Wrappedafterbreak}
            {\kern\fontdimen2\font}%
        }%
        
        % Allow breaks at special characters using \PYG... macros.
        \Wrappedbreaksatspecials
        % Breaks at punctuation characters . , ; ? ! and / need catcode=\active 	
        \OriginalVerbatim[#1,codes*=\Wrappedbreaksatpunct]%
    }
    \makeatother

    % Exact colors from NB
    \definecolor{incolor}{HTML}{303F9F}
    \definecolor{outcolor}{HTML}{D84315}
    \definecolor{cellborder}{HTML}{CFCFCF}
    \definecolor{cellbackground}{HTML}{F7F7F7}
    
    % prompt
    \makeatletter
    \newcommand{\boxspacing}{\kern\kvtcb@left@rule\kern\kvtcb@boxsep}
    \makeatother
    \newcommand{\prompt}[4]{
        \ttfamily\llap{{\color{#2}[#3]:\hspace{3pt}#4}}\vspace{-\baselineskip}
    }
    

    
    % Prevent overflowing lines due to hard-to-break entities
    \sloppy 
    % Setup hyperref package
    \hypersetup{
      breaklinks=true,  % so long urls are correctly broken across lines
      colorlinks=true,
      urlcolor=urlcolor,
      linkcolor=linkcolor,
      citecolor=citecolor,
      }
    % Slightly bigger margins than the latex defaults
    
    \geometry{verbose,tmargin=1in,bmargin=1in,lmargin=1in,rmargin=1in}
    
    

\begin{document}
    

    
    

    \setcounter{secnumdepth}{0}
\title{Index Funds vs. Income Funds for Yearly Income: An Analysis of Resiliency over a Ten Year Period }
\author{Alexei Marcilio}
\date{\today}
\maketitle
\thispagestyle{empty}
\newpage
\setcounter{page}{1} 
\tableofcontents
\newpage
\pagenumbering{arabic}
    \hypertarget{introduction}{%
\section{Introduction}\label{introduction}}

    Is it better to draw capital gains from a market ETF (Exchange Traded
Fund) or use a monthly income ETF? Also, which ETFs provide monthly
income yet still appreciate over time? These are questions on the mind
of many investors, especially those who are retired and are counting on
the income generated from their portfolio. The purpose of this report is
to answer these questions.

Every investor is looking for yield, and there are many ETFs that
provide a monthly dividend. Some of these dividends however seem too
good to be true, and they are. If the dividends paid by an ETF are
greater than the dividends of the funds held then the yield is partly
made up of ``return of capital''.

Would a Canadian investor who needs monthly income be better off buying
an ETF that pays a high dividend, or would he or she be better off
buying a market ETF and selling shares to obtain the income? In order to
find out I will compare six of the most popular ETFs in Canada. I'll
draw the same amount from each fund each year and see how much of the
fund remains after a ten year period.

    \hypertarget{methods}{%
\section{Methods}\label{methods}}

    Python will be used to compare the ETFs. The closing price and dividends
for each of these stocks was obtained with the Yahoo Finance API. I've
written some functions to buy stock if the dividend exceeds our income
needs, and sell stock if it falls short.

First I will start with some typical analysis. I will look at the price
changes over time, the volatility, and how well these ETFs correlate to
each other.

    \hypertarget{stocks}{%
\subsection{Stocks}\label{stocks}}

    The following table shows the stocks used in this analysis. They include
the perennial favorite XIU, which is the first and largest Canadian ETF,
and CDZ, the Canadian Dividend Aristocrats Index, an ETF made up of a
diversified group of high quality Canadian dividend companies.

\scalebox{0.8}{
\begin{tabular}{ll}
\toprule
ETF &  Description \\
\midrule
ZEB &    BMO Equal Weight Banks Index ETF \\
FIE &    Common and preferred shares, corporate bonds and income trust units 
in the Canadian financial sector \\
CDZ &    Diversified exposure to a portfolio of high quality Canadian dividend paying companies \\
XTR &    Income-bearing assets, including equities, fixed income securities and real estate investment trusts \\
XDV &    Exposure to 30 of the highest yielding Canadian companies in the Dow Jones Canada Total Market Index \\
XIU &    Exposure to large, established Canadian companies (Top 60 in the TSX) \\
\bottomrule
\end{tabular}
}
    \begin{tcolorbox}[breakable, size=fbox, boxrule=1pt, pad at break*=1mm,colback=cellbackground, colframe=cellborder]
\prompt{In}{incolor}{782}{\boxspacing}
\begin{Verbatim}[commandchars=\\\{\}]
\PY{c+c1}{\PYZsh{} Import libraries}
\PY{k+kn}{from} \PY{n+nn}{pandas\PYZus{}datareader} \PY{k+kn}{import} \PY{n}{data} \PY{k}{as} \PY{n}{wb}
\PY{k+kn}{import} \PY{n+nn}{pandas} \PY{k}{as} \PY{n+nn}{pd}
\PY{k+kn}{import} \PY{n+nn}{matplotlib}\PY{n+nn}{.}\PY{n+nn}{pyplot} \PY{k}{as} \PY{n+nn}{plt}
\PY{k+kn}{import} \PY{n+nn}{seaborn} \PY{k}{as} \PY{n+nn}{sns}
\PY{k+kn}{from} \PY{n+nn}{datetime} \PY{k+kn}{import} \PY{n}{datetime}
\PY{k+kn}{import} \PY{n+nn}{numpy} \PY{k}{as} \PY{n+nn}{np}
\PY{o}{\PYZpc{}}\PY{k}{matplotlib} inline
\end{Verbatim}
\end{tcolorbox}

    \begin{tcolorbox}[breakable, size=fbox, boxrule=1pt, pad at break*=1mm,colback=cellbackground, colframe=cellborder]
\prompt{In}{incolor}{783}{\boxspacing}
\begin{Verbatim}[commandchars=\\\{\}]
\PY{k+kn}{import} \PY{n+nn}{warnings}
\PY{n}{warnings}\PY{o}{.}\PY{n}{filterwarnings}\PY{p}{(}\PY{l+s+s1}{\PYZsq{}}\PY{l+s+s1}{ignore}\PY{l+s+s1}{\PYZsq{}}\PY{p}{)}
\end{Verbatim}
\end{tcolorbox}

    \begin{tcolorbox}[breakable, size=fbox, boxrule=1pt, pad at break*=1mm,colback=cellbackground, colframe=cellborder]
\prompt{In}{incolor}{784}{\boxspacing}
\begin{Verbatim}[commandchars=\\\{\}]
\PY{c+c1}{\PYZsh{} Start here \PYZhy{} check these tickers}
\PY{n}{tickers} \PY{o}{=} \PY{p}{[}\PY{l+s+s1}{\PYZsq{}}\PY{l+s+s1}{XIU.TO}\PY{l+s+s1}{\PYZsq{}}\PY{p}{,}\PY{l+s+s1}{\PYZsq{}}\PY{l+s+s1}{CDZ.TO}\PY{l+s+s1}{\PYZsq{}}\PY{p}{,}\PY{l+s+s1}{\PYZsq{}}\PY{l+s+s1}{XDV.TO}\PY{l+s+s1}{\PYZsq{}}\PY{p}{,}\PY{l+s+s1}{\PYZsq{}}\PY{l+s+s1}{XTR.TO}\PY{l+s+s1}{\PYZsq{}}\PY{p}{,}\PY{l+s+s1}{\PYZsq{}}\PY{l+s+s1}{ZEB.TO}\PY{l+s+s1}{\PYZsq{}}\PY{p}{,}\PY{l+s+s1}{\PYZsq{}}\PY{l+s+s1}{FIE.TO}\PY{l+s+s1}{\PYZsq{}}\PY{p}{]}
\PY{c+c1}{\PYZsh{} Years of interest}
\PY{n}{years} \PY{o}{=} \PY{p}{[}\PY{l+s+s1}{\PYZsq{}}\PY{l+s+s1}{2011}\PY{l+s+s1}{\PYZsq{}}\PY{p}{,}\PY{l+s+s1}{\PYZsq{}}\PY{l+s+s1}{2012}\PY{l+s+s1}{\PYZsq{}}\PY{p}{,}\PY{l+s+s1}{\PYZsq{}}\PY{l+s+s1}{2013}\PY{l+s+s1}{\PYZsq{}}\PY{p}{,}\PY{l+s+s1}{\PYZsq{}}\PY{l+s+s1}{2014}\PY{l+s+s1}{\PYZsq{}}\PY{p}{,}\PY{l+s+s1}{\PYZsq{}}\PY{l+s+s1}{2015}\PY{l+s+s1}{\PYZsq{}}\PY{p}{,}\PY{l+s+s1}{\PYZsq{}}\PY{l+s+s1}{2016}\PY{l+s+s1}{\PYZsq{}}\PY{p}{,}\PY{l+s+s1}{\PYZsq{}}\PY{l+s+s1}{2017}\PY{l+s+s1}{\PYZsq{}}\PY{p}{,}\PY{l+s+s1}{\PYZsq{}}\PY{l+s+s1}{2018}\PY{l+s+s1}{\PYZsq{}}\PY{p}{,}\PY{l+s+s1}{\PYZsq{}}\PY{l+s+s1}{2020}\PY{l+s+s1}{\PYZsq{}}\PY{p}{]}
\end{Verbatim}
\end{tcolorbox}

    \begin{tcolorbox}[breakable, size=fbox, boxrule=1pt, pad at break*=1mm,colback=cellbackground, colframe=cellborder]
\prompt{In}{incolor}{785}{\boxspacing}
\begin{Verbatim}[commandchars=\\\{\}]
\PY{c+c1}{\PYZsh{}Pull in yahoo finance data}
\PY{n}{new\PYZus{}data} \PY{o}{=} \PY{n}{pd}\PY{o}{.}\PY{n}{DataFrame}\PY{p}{(}\PY{p}{)}
\PY{k}{for} \PY{n}{t} \PY{o+ow}{in} \PY{n}{tickers}\PY{p}{:}
    \PY{n}{new\PYZus{}data}\PY{p}{[}\PY{n}{t}\PY{p}{]} \PY{o}{=} \PY{n}{wb}\PY{o}{.}\PY{n}{DataReader}\PY{p}{(}\PY{n}{t}\PY{p}{,} \PY{n}{data\PYZus{}source}\PY{o}{=}\PY{l+s+s1}{\PYZsq{}}\PY{l+s+s1}{yahoo}\PY{l+s+s1}{\PYZsq{}}\PY{p}{,} \PY{n}{start}\PY{o}{=}\PY{l+s+s1}{\PYZsq{}}\PY{l+s+s1}{2011\PYZhy{}1\PYZhy{}1}\PY{l+s+s1}{\PYZsq{}}\PY{p}{)}\PY{p}{[}\PY{l+s+s1}{\PYZsq{}}\PY{l+s+s1}{Close}\PY{l+s+s1}{\PYZsq{}}\PY{p}{]}
\end{Verbatim}
\end{tcolorbox}

    \begin{tcolorbox}[breakable, size=fbox, boxrule=1pt, pad at break*=1mm,colback=cellbackground, colframe=cellborder]
\prompt{In}{incolor}{786}{\boxspacing}
\begin{Verbatim}[commandchars=\\\{\}]
\PY{n}{new\PYZus{}data}\PY{p}{[}\PY{l+s+s1}{\PYZsq{}}\PY{l+s+s1}{trade\PYZus{}year}\PY{l+s+s1}{\PYZsq{}}\PY{p}{]} \PY{o}{=} \PY{n}{new\PYZus{}data}\PY{o}{.}\PY{n}{index}\PY{o}{.}\PY{n}{map}\PY{p}{(}\PY{n+nb}{str}\PY{p}{)}\PY{o}{.}\PY{n}{str}\PY{o}{.}\PY{n}{slice}\PY{p}{(}\PY{l+m+mi}{0}\PY{p}{,} \PY{l+m+mi}{4}\PY{p}{)}
\end{Verbatim}
\end{tcolorbox}

    \hypertarget{results}{%
\section{Results}\label{results}}

    \hypertarget{visualization}{%
\subsection{Visualization}\label{visualization}}

    Let's visualize these ETFs to see how they compare with regards to
return, price changes, volatility and correlation.

    \hypertarget{time-series}{%
\subsubsection{Time Series}\label{time-series}}

    

    Let's take a look at how the prices have changed for these stocks over
the last decade. The graph below shows a similar pattern among 4 of the
stocks. FIE and XTR are the outliers with what seem like much less of an
increase in price over the decade. This makes sense as a large part of
these two ETFs are made up of bonds and preferred shares.

ZEB seems to be the clear winner. The similar pattern in the top 4
stocks makes sense as there's obviously overlap among them. ZEB is an
equal weighted index of Canadian banks, and XIU and other dividend
stocks will of course include Canadian banks, which are favorites among
dividend income seekers.

    \begin{tcolorbox}[breakable, size=fbox, boxrule=1pt, pad at break*=1mm,colback=cellbackground, colframe=cellborder]
\prompt{In}{incolor}{787}{\boxspacing}
\begin{Verbatim}[commandchars=\\\{\}]
\PY{k}{for} \PY{n}{t} \PY{o+ow}{in} \PY{n}{tickers}\PY{p}{:}
    \PY{n}{new\PYZus{}data}\PY{p}{[}\PY{n}{t}\PY{p}{]}\PY{o}{.}\PY{n}{plot}\PY{p}{(}\PY{n}{legend}\PY{o}{=}\PY{k+kc}{True}\PY{p}{,} \PY{n}{figsize}\PY{o}{=}\PY{p}{(}\PY{l+m+mi}{10}\PY{p}{,} \PY{l+m+mi}{6}\PY{p}{)}\PY{p}{,} \PYZbs{}
    \PY{n}{title}\PY{o}{=}\PY{l+s+s1}{\PYZsq{}}\PY{l+s+s1}{Closing Price}\PY{l+s+s1}{\PYZsq{}}\PY{p}{,} \PYZbs{}
    \PY{n}{label}\PY{o}{=}\PY{n}{t}\PY{p}{)}
\end{Verbatim}
\end{tcolorbox}

    \begin{center}
    \adjustimage{max size={0.9\linewidth}{0.9\paperheight}}{output_19_0.png}
    \end{center}
    { \hspace*{\fill} \\}
    
    The stock prices vary tremendously, for example CDZ is well over \$30 a
share and FIE is about \$7 at the time this was written it makes sense
to normalize all the prices to 100. This will make it easier to compare
the price changes over time.

\[
\frac{P_t}{P_0} * 100
\]

The normalized plot below further emphasizes how well ZEB, BMO's equally
weighted bank index, has done over the last decade.

    \begin{tcolorbox}[breakable, size=fbox, boxrule=1pt, pad at break*=1mm,colback=cellbackground, colframe=cellborder]
\prompt{In}{incolor}{809}{\boxspacing}
\begin{Verbatim}[commandchars=\\\{\}]
\PY{c+c1}{\PYZsh{} We have to exclude the trade\PYZus{}year column as this is a string.}
\PY{n}{df\PYZus{}no\PYZus{}date} \PY{o}{=} \PY{n}{new\PYZus{}data}\PY{o}{.}\PY{n}{iloc}\PY{p}{[}\PY{p}{:}\PY{p}{,}\PY{l+m+mi}{0}\PY{p}{:}\PY{l+m+mi}{6}\PY{p}{]}

\PY{c+c1}{\PYZsh{} Now we normalize the data}
\PY{p}{(}\PY{n}{df\PYZus{}no\PYZus{}date}\PY{o}{.}\PY{n}{iloc}\PY{p}{[}\PY{p}{:}\PY{p}{,}\PY{l+m+mi}{0}\PY{p}{:}\PY{l+m+mi}{6}\PY{p}{]} \PY{o}{/} \PY{n}{df\PYZus{}no\PYZus{}date}\PY{o}{.}\PY{n}{iloc}\PY{p}{[}\PY{l+m+mi}{0}\PY{p}{]} \PY{o}{*} \PY{l+m+mi}{100}\PY{p}{)}\PY{o}{.}\PY{n}{plot}\PY{p}{(}\PY{n}{figsize} \PY{o}{=} \PY{p}{(}\PY{l+m+mi}{15}\PY{p}{,} \PY{l+m+mi}{6}\PY{p}{)}\PY{p}{)}\PY{p}{;}
\PY{n}{plt}\PY{o}{.}\PY{n}{show}\PY{p}{(}\PY{p}{)}
\end{Verbatim}
\end{tcolorbox}

    \begin{center}
    \adjustimage{max size={0.9\linewidth}{0.9\paperheight}}{output_21_0.png}
    \end{center}
    { \hspace*{\fill} \\}
    
    \hypertarget{returns-by-year}{%
\subsubsection{Returns by Year}\label{returns-by-year}}

    \begin{tcolorbox}[breakable, size=fbox, boxrule=1pt, pad at break*=1mm,colback=cellbackground, colframe=cellborder]
\prompt{In}{incolor}{789}{\boxspacing}
\begin{Verbatim}[commandchars=\\\{\}]
\PY{n}{new\PYZus{}data}\PY{o}{.}\PY{n}{index} \PY{o}{=} \PY{n}{new\PYZus{}data}\PY{o}{.}\PY{n}{index}\PY{o}{.}\PY{n}{map}\PY{p}{(}\PY{n+nb}{str}\PY{p}{)}

\PY{c+c1}{\PYZsh{} Add a date column}
\PY{n}{new\PYZus{}data}\PY{p}{[}\PY{l+s+s1}{\PYZsq{}}\PY{l+s+s1}{trade\PYZus{}year}\PY{l+s+s1}{\PYZsq{}}\PY{p}{]} \PY{o}{=} \PY{n}{new\PYZus{}data}\PY{o}{.}\PY{n}{index}\PY{o}{.}\PY{n}{str}\PY{o}{.}\PY{n}{slice}\PY{p}{(}\PY{l+m+mi}{0}\PY{p}{,} \PY{l+m+mi}{4}\PY{p}{)}

\PY{c+c1}{\PYZsh{} Prevent warning when writing back to object}
\PY{n}{pd}\PY{o}{.}\PY{n}{options}\PY{o}{.}\PY{n}{mode}\PY{o}{.}\PY{n}{chained\PYZus{}assignment} \PY{o}{=} \PY{k+kc}{None}  \PY{c+c1}{\PYZsh{} default=\PYZsq{}warn\PYZsq{}}

\PY{c+c1}{\PYZsh{} A list used to keep the stock return data}
\PY{n}{values} \PY{o}{=} \PY{p}{[}\PY{p}{]}

\PY{c+c1}{\PYZsh{} Loop through each year and the each ticker and find the last day and the first day}
\PY{c+c1}{\PYZsh{} of the year to calculate the return}
\PY{k}{for} \PY{n}{y} \PY{o+ow}{in} \PY{n}{years}\PY{p}{:}
    \PY{k}{for} \PY{n}{t} \PY{o+ow}{in} \PY{n}{tickers}\PY{p}{:}
        \PY{c+c1}{\PYZsh{} We can use iloc to find the last value and the first value}
        \PY{n}{rate\PYZus{}return} \PY{o}{=} \PY{n}{new\PYZus{}data}\PY{p}{[}\PY{n}{new\PYZus{}data}\PY{p}{[}\PY{l+s+s1}{\PYZsq{}}\PY{l+s+s1}{trade\PYZus{}year}\PY{l+s+s1}{\PYZsq{}}\PY{p}{]} \PY{o}{==} \PY{n}{y}\PY{p}{]}\PY{o}{.}\PY{n}{iloc}\PY{p}{[}\PY{o}{\PYZhy{}}\PY{l+m+mi}{1}\PY{p}{]}\PY{p}{[}\PY{n}{t}\PY{p}{]} \PY{o}{/} \PYZbs{}
        \PY{n}{new\PYZus{}data}\PY{p}{[}\PY{n}{new\PYZus{}data}\PY{p}{[}\PY{l+s+s1}{\PYZsq{}}\PY{l+s+s1}{trade\PYZus{}year}\PY{l+s+s1}{\PYZsq{}}\PY{p}{]} \PY{o}{==} \PY{n}{y}\PY{p}{]}\PY{o}{.}\PY{n}{iloc}\PY{p}{[}\PY{l+m+mi}{0}\PY{p}{]}\PY{p}{[}\PY{n}{t}\PY{p}{]} \PY{o}{\PYZhy{}} \PY{l+m+mi}{1}
        \PY{n}{values}\PY{o}{.}\PY{n}{append}\PY{p}{(}\PY{p}{[}\PY{n}{t}\PY{p}{,} \PY{n}{y}\PY{p}{,} \PY{n}{rate\PYZus{}return}\PY{p}{]}\PY{p}{)}
    

\PY{c+c1}{\PYZsh{} Create the DataFrame from the list}
\PY{n}{values\PYZus{}df} \PY{o}{=} \PY{n}{pd}\PY{o}{.}\PY{n}{DataFrame}\PY{p}{(}\PY{n}{values}\PY{p}{)}  
\PY{c+c1}{\PYZsh{} Rename the DataFrame columns}
\PY{n}{values\PYZus{}df}\PY{o}{.}\PY{n}{columns} \PY{o}{=} \PY{p}{[}\PY{l+s+s1}{\PYZsq{}}\PY{l+s+s1}{ticker}\PY{l+s+s1}{\PYZsq{}}\PY{p}{,}\PY{l+s+s1}{\PYZsq{}}\PY{l+s+s1}{trade\PYZus{}year}\PY{l+s+s1}{\PYZsq{}}\PY{p}{,}\PY{l+s+s1}{\PYZsq{}}\PY{l+s+s1}{yr\PYZus{}return}\PY{l+s+s1}{\PYZsq{}}\PY{p}{]}
\PY{c+c1}{\PYZsh{} Sort the DataFrame}
\PY{n}{values\PYZus{}df}\PY{o}{.}\PY{n}{sort\PYZus{}values}\PY{p}{(}\PY{n}{by}\PY{o}{=}\PY{p}{[}\PY{l+s+s1}{\PYZsq{}}\PY{l+s+s1}{ticker}\PY{l+s+s1}{\PYZsq{}}\PY{p}{,}\PY{l+s+s1}{\PYZsq{}}\PY{l+s+s1}{trade\PYZus{}year}\PY{l+s+s1}{\PYZsq{}}\PY{p}{]}\PY{p}{,} \PY{n}{inplace}\PY{o}{=}\PY{k+kc}{True}\PY{p}{)}
\PY{c+c1}{\PYZsh{} Format the value of the yr\PYZus{}return column}
\PY{n}{values\PYZus{}df}\PY{p}{[}\PY{l+s+s1}{\PYZsq{}}\PY{l+s+s1}{yr\PYZus{}return}\PY{l+s+s1}{\PYZsq{}}\PY{p}{]} \PY{o}{=} \PY{n+nb}{round}\PY{p}{(}\PY{n}{values\PYZus{}df}\PY{o}{.}\PY{n}{yr\PYZus{}return}\PY{p}{,}\PY{l+m+mi}{4}\PY{p}{)} \PY{o}{*} \PY{l+m+mi}{100}

\PY{c+c1}{\PYZsh{} Make the index just the date instead of the datetime}
\PY{n}{new\PYZus{}data}\PY{o}{.}\PY{n}{index} \PY{o}{=} \PY{n}{new\PYZus{}data}\PY{o}{.}\PY{n}{index}\PY{o}{.}\PY{n}{str}\PY{o}{.}\PY{n}{slice}\PY{p}{(}\PY{l+m+mi}{0}\PY{p}{,}\PY{l+m+mi}{10}\PY{p}{)}
\end{Verbatim}
\end{tcolorbox}

    The following figure shows the returns for each of the ETFs by year.
There have been three or four down years for each of the funds in the
last decade.

ZEB has had a couple of stellar years.

    \begin{tcolorbox}[breakable, size=fbox, boxrule=1pt, pad at break*=1mm,colback=cellbackground, colframe=cellborder]
\prompt{In}{incolor}{791}{\boxspacing}
\begin{Verbatim}[commandchars=\\\{\}]
\PY{n}{values\PYZus{}df}

\PY{n}{g} \PY{o}{=} \PY{n}{sns}\PY{o}{.}\PY{n}{catplot}\PY{p}{(}\PY{n}{x}\PY{o}{=}\PY{l+s+s1}{\PYZsq{}}\PY{l+s+s1}{ticker}\PY{l+s+s1}{\PYZsq{}}\PY{p}{,} \PY{n}{y}\PY{o}{=}\PY{l+s+s1}{\PYZsq{}}\PY{l+s+s1}{yr\PYZus{}return}\PY{l+s+s1}{\PYZsq{}}\PY{p}{,} \PY{n}{hue}\PY{o}{=}\PY{l+s+s1}{\PYZsq{}}\PY{l+s+s1}{trade\PYZus{}year}\PY{l+s+s1}{\PYZsq{}}\PY{p}{,} \PY{n}{data}\PY{o}{=}\PY{n}{values\PYZus{}df}\PY{p}{,} \PY{n}{kind}\PY{o}{=}\PY{l+s+s1}{\PYZsq{}}\PY{l+s+s1}{bar}\PY{l+s+s1}{\PYZsq{}}\PY{p}{,} \PY{n}{legend}\PY{o}{=}\PY{k+kc}{False}\PY{p}{)}
\PY{n}{g}\PY{o}{.}\PY{n}{fig}\PY{o}{.}\PY{n}{set\PYZus{}figwidth}\PY{p}{(}\PY{l+m+mi}{9}\PY{p}{)}
\PY{n}{g}\PY{o}{.}\PY{n}{fig}\PY{o}{.}\PY{n}{suptitle}\PY{p}{(}\PY{l+s+s2}{\PYZdq{}}\PY{l+s+s2}{Year by Year Return}\PY{l+s+s2}{\PYZdq{}}\PY{p}{)}
\PY{n}{g}\PY{o}{.}\PY{n}{set\PYZus{}axis\PYZus{}labels}\PY{p}{(}\PY{l+s+s2}{\PYZdq{}}\PY{l+s+s2}{Stock}\PY{l+s+s2}{\PYZdq{}}\PY{p}{,} \PY{l+s+s2}{\PYZdq{}}\PY{l+s+s2}{Percent Return}\PY{l+s+s2}{\PYZdq{}}\PY{p}{)}\PY{p}{;}
\end{Verbatim}
\end{tcolorbox}

    \begin{center}
    \adjustimage{max size={0.9\linewidth}{0.9\paperheight}}{output_25_0.png}
    \end{center}
    { \hspace*{\fill} \\}
    
    \hypertarget{volatility}{%
\subsubsection{Volatility}\label{volatility}}

    The following plots show the distributions of the prices for each of
these tickers over the decade. XTR shows the least amount of volatility.
This makes sense due to its high bond component.

    \begin{tcolorbox}[breakable, size=fbox, boxrule=1pt, pad at break*=1mm,colback=cellbackground, colframe=cellborder]
\prompt{In}{incolor}{773}{\boxspacing}
\begin{Verbatim}[commandchars=\\\{\}]
\PY{k}{for} \PY{n}{t} \PY{o+ow}{in} \PY{n}{tickers}\PY{p}{:}
    \PY{n}{plt}\PY{o}{.}\PY{n}{figure}\PY{p}{(}\PY{n}{figsize}\PY{o}{=}\PY{p}{(}\PY{l+m+mi}{6}\PY{p}{,}\PY{l+m+mf}{2.2}\PY{p}{)}\PY{p}{)}
    \PY{c+c1}{\PYZsh{}fig, axs = plt.subplots(2, 3)}
    \PY{n}{sns}\PY{o}{.}\PY{n}{distplot}\PY{p}{(}\PY{n}{new\PYZus{}data}\PY{p}{[}\PY{n}{t}\PY{p}{]}\PY{o}{.}\PY{n}{dropna}\PY{p}{(}\PY{p}{)}\PY{p}{,} \PY{n}{bins}\PY{o}{=}\PY{l+m+mi}{50}\PY{p}{,} \PY{n}{color}\PY{o}{=}\PY{l+s+s1}{\PYZsq{}}\PY{l+s+s1}{purple}\PY{l+s+s1}{\PYZsq{}}\PY{p}{)}\PY{p}{;}
\end{Verbatim}
\end{tcolorbox}

    \begin{center}
    \adjustimage{max size={0.9\linewidth}{0.9\paperheight}}{output_28_0.png}
    \end{center}
    { \hspace*{\fill} \\}
    
    \begin{center}
    \adjustimage{max size={0.9\linewidth}{0.9\paperheight}}{output_28_1.png}
    \end{center}
    { \hspace*{\fill} \\}
    
    \begin{center}
    \adjustimage{max size={0.9\linewidth}{0.9\paperheight}}{output_28_2.png}
    \end{center}
    { \hspace*{\fill} \\}
    
    \begin{center}
    \adjustimage{max size={0.9\linewidth}{0.9\paperheight}}{output_28_3.png}
    \end{center}
    { \hspace*{\fill} \\}
    
    \begin{center}
    \adjustimage{max size={0.9\linewidth}{0.9\paperheight}}{output_28_4.png}
    \end{center}
    { \hspace*{\fill} \\}
    
    \begin{center}
    \adjustimage{max size={0.9\linewidth}{0.9\paperheight}}{output_28_5.png}
    \end{center}
    { \hspace*{\fill} \\}
    
    \hypertarget{correlations}{%
\subsubsection{Correlations}\label{correlations}}

    The heatmap below shows how correlated the ETFs are. It's no surprise
that the pure stock ETFs are highly correlated.

CDZ and XDV are highly correlated (94\%). This is understandable as they
both hold the major banks. Again XTR is the outlier here as it includes
some US stock and it's weighted about 50/50 between stocks and bonds.

    \begin{tcolorbox}[breakable, size=fbox, boxrule=1pt, pad at break*=1mm,colback=cellbackground, colframe=cellborder]
\prompt{In}{incolor}{792}{\boxspacing}
\begin{Verbatim}[commandchars=\\\{\}]
\PY{c+c1}{\PYZsh{} Calculate correlations}
\PY{n}{corr} \PY{o}{=} \PY{n}{new\PYZus{}data}\PY{p}{[}\PY{n}{tickers}\PY{p}{]}\PY{o}{.}\PY{n}{corr}\PY{p}{(}\PY{p}{)}
\PY{c+c1}{\PYZsh{} Heatmap}
\PY{n}{sns}\PY{o}{.}\PY{n}{heatmap}\PY{p}{(}\PY{n}{corr}\PY{p}{,} \PY{n}{cmap} \PY{o}{=} \PY{l+s+s2}{\PYZdq{}}\PY{l+s+s2}{Blues}\PY{l+s+s2}{\PYZdq{}}\PY{p}{,} \PY{n}{annot}\PY{o}{=}\PY{k+kc}{True}\PY{p}{)}\PY{p}{;}
\end{Verbatim}
\end{tcolorbox}

    \begin{center}
    \adjustimage{max size={0.9\linewidth}{0.9\paperheight}}{output_31_0.png}
    \end{center}
    { \hspace*{\fill} \\}
    
    \hypertarget{income---which-is-best}{%
\subsection{Income - Which is Best?}\label{income---which-is-best}}

    To answer the question of which ETF is the most valuable after a decade
of providing income we need to create a dataframe of the start and end
closing prices for each stock, and the sum of all the dividends provided
for each year.

The following code creates such a dataframe using data accessed from
Yahoo Finance's API.

    \hypertarget{feature-engineering}{%
\subsubsection{Feature Engineering}\label{feature-engineering}}

    \begin{tcolorbox}[breakable, size=fbox, boxrule=1pt, pad at break*=1mm,colback=cellbackground, colframe=cellborder]
\prompt{In}{incolor}{751}{\boxspacing}
\begin{Verbatim}[commandchars=\\\{\}]
\PY{c+c1}{\PYZsh{} Let\PYZsq{}s create a table for the yearly data}
\PY{n}{dfReturns} \PY{o}{=} \PY{n}{pd}\PY{o}{.}\PY{n}{DataFrame}\PY{p}{(}\PY{p}{\PYZob{}}\PY{l+s+s2}{\PYZdq{}}\PY{l+s+s2}{Year}\PY{l+s+s2}{\PYZdq{}} \PY{p}{:}\PY{p}{[}\PY{l+m+mi}{2011}\PY{p}{,}\PY{l+m+mi}{2012}\PY{p}{,}\PY{l+m+mi}{2013}\PY{p}{,}\PY{l+m+mi}{2014}\PY{p}{,}\PY{l+m+mi}{2015}\PY{p}{,}\PY{l+m+mi}{2016}\PY{p}{,}\PY{l+m+mi}{2017}\PY{p}{,}\PY{l+m+mi}{2018}\PY{p}{,}\PY{l+m+mi}{2019}\PY{p}{,}\PY{l+m+mi}{2020}\PY{p}{,}\PY{l+m+mi}{2021}\PY{p}{]}\PY{p}{\PYZcb{}}\PY{p}{)}
\end{Verbatim}
\end{tcolorbox}

    \begin{tcolorbox}[breakable, size=fbox, boxrule=1pt, pad at break*=1mm,colback=cellbackground, colframe=cellborder]
\prompt{In}{incolor}{752}{\boxspacing}
\begin{Verbatim}[commandchars=\\\{\}]
\PY{c+c1}{\PYZsh{}new\PYZus{}data[\PYZsq{}Date\PYZsq{}]= pd.to\PYZus{}datetime(new\PYZus{}data[\PYZsq{}Date\PYZsq{}])}
\PY{n}{new\PYZus{}data}\PY{o}{.}\PY{n}{index} \PY{o}{=} \PY{n}{pd}\PY{o}{.}\PY{n}{to\PYZus{}datetime}\PY{p}{(}\PY{n}{new\PYZus{}data}\PY{o}{.}\PY{n}{index}\PY{p}{)}
\end{Verbatim}
\end{tcolorbox}

    \hypertarget{retrieve-dividend-data}{%
\subsubsection{Retrieve Dividend Data}\label{retrieve-dividend-data}}

    \begin{tcolorbox}[breakable, size=fbox, boxrule=1pt, pad at break*=1mm,colback=cellbackground, colframe=cellborder]
\prompt{In}{incolor}{753}{\boxspacing}
\begin{Verbatim}[commandchars=\\\{\}]
\PY{c+c1}{\PYZsh{} Yahoo finance app}
\PY{k+kn}{import} \PY{n+nn}{yfinance} \PY{k}{as} \PY{n+nn}{yf}
\end{Verbatim}
\end{tcolorbox}

    \begin{tcolorbox}[breakable, size=fbox, boxrule=1pt, pad at break*=1mm,colback=cellbackground, colframe=cellborder]
\prompt{In}{incolor}{754}{\boxspacing}
\begin{Verbatim}[commandchars=\\\{\}]
\PY{c+c1}{\PYZsh{} bring in the data}
\PY{n}{cdz} \PY{o}{=} \PY{n}{yf}\PY{o}{.}\PY{n}{Ticker}\PY{p}{(}\PY{l+s+s2}{\PYZdq{}}\PY{l+s+s2}{CDZ.TO}\PY{l+s+s2}{\PYZdq{}}\PY{p}{)}
\PY{n}{xiu} \PY{o}{=} \PY{n}{yf}\PY{o}{.}\PY{n}{Ticker}\PY{p}{(}\PY{l+s+s2}{\PYZdq{}}\PY{l+s+s2}{XIU.TO}\PY{l+s+s2}{\PYZdq{}}\PY{p}{)}
\PY{n}{xdv} \PY{o}{=} \PY{n}{yf}\PY{o}{.}\PY{n}{Ticker}\PY{p}{(}\PY{l+s+s2}{\PYZdq{}}\PY{l+s+s2}{XDV.TO}\PY{l+s+s2}{\PYZdq{}}\PY{p}{)}
\PY{n}{xtr} \PY{o}{=} \PY{n}{yf}\PY{o}{.}\PY{n}{Ticker}\PY{p}{(}\PY{l+s+s2}{\PYZdq{}}\PY{l+s+s2}{XTR.TO}\PY{l+s+s2}{\PYZdq{}}\PY{p}{)}
\PY{n}{zeb} \PY{o}{=} \PY{n}{yf}\PY{o}{.}\PY{n}{Ticker}\PY{p}{(}\PY{l+s+s2}{\PYZdq{}}\PY{l+s+s2}{ZEB.TO}\PY{l+s+s2}{\PYZdq{}}\PY{p}{)}
\PY{n}{fie} \PY{o}{=} \PY{n}{yf}\PY{o}{.}\PY{n}{Ticker}\PY{p}{(}\PY{l+s+s2}{\PYZdq{}}\PY{l+s+s2}{FIE.TO}\PY{l+s+s2}{\PYZdq{}}\PY{p}{)}
\end{Verbatim}
\end{tcolorbox}

    \begin{tcolorbox}[breakable, size=fbox, boxrule=1pt, pad at break*=1mm,colback=cellbackground, colframe=cellborder]
\prompt{In}{incolor}{755}{\boxspacing}
\begin{Verbatim}[commandchars=\\\{\}]
\PY{c+c1}{\PYZsh{} Create an 11 year period}
\PY{n}{CDZ\PYZus{}hist} \PY{o}{=} \PY{n}{cdz}\PY{o}{.}\PY{n}{history}\PY{p}{(}\PY{n}{period}\PY{o}{=}\PY{l+s+s2}{\PYZdq{}}\PY{l+s+s2}{11y}\PY{l+s+s2}{\PYZdq{}}\PY{p}{)}
\PY{n}{XIU\PYZus{}hist} \PY{o}{=} \PY{n}{xiu}\PY{o}{.}\PY{n}{history}\PY{p}{(}\PY{n}{period}\PY{o}{=}\PY{l+s+s2}{\PYZdq{}}\PY{l+s+s2}{11y}\PY{l+s+s2}{\PYZdq{}}\PY{p}{)}
\PY{n}{XDV\PYZus{}hist} \PY{o}{=} \PY{n}{xdv}\PY{o}{.}\PY{n}{history}\PY{p}{(}\PY{n}{period}\PY{o}{=}\PY{l+s+s2}{\PYZdq{}}\PY{l+s+s2}{11y}\PY{l+s+s2}{\PYZdq{}}\PY{p}{)}
\PY{n}{XTR\PYZus{}hist} \PY{o}{=} \PY{n}{cdz}\PY{o}{.}\PY{n}{history}\PY{p}{(}\PY{n}{period}\PY{o}{=}\PY{l+s+s2}{\PYZdq{}}\PY{l+s+s2}{11y}\PY{l+s+s2}{\PYZdq{}}\PY{p}{)}
\PY{n}{ZEB\PYZus{}hist} \PY{o}{=} \PY{n}{zeb}\PY{o}{.}\PY{n}{history}\PY{p}{(}\PY{n}{period}\PY{o}{=}\PY{l+s+s2}{\PYZdq{}}\PY{l+s+s2}{11y}\PY{l+s+s2}{\PYZdq{}}\PY{p}{)}
\PY{n}{FIE\PYZus{}hist} \PY{o}{=} \PY{n}{fie}\PY{o}{.}\PY{n}{history}\PY{p}{(}\PY{n}{period}\PY{o}{=}\PY{l+s+s2}{\PYZdq{}}\PY{l+s+s2}{11y}\PY{l+s+s2}{\PYZdq{}}\PY{p}{)}
\end{Verbatim}
\end{tcolorbox}

    \hypertarget{join-tables}{%
\subsubsection{Join Tables}\label{join-tables}}

    \begin{tcolorbox}[breakable, size=fbox, boxrule=1pt, pad at break*=1mm,colback=cellbackground, colframe=cellborder]
\prompt{In}{incolor}{756}{\boxspacing}
\begin{Verbatim}[commandchars=\\\{\}]
\PY{c+c1}{\PYZsh{} Add the dividend data for each stock to the main table}
\PY{n}{new\PYZus{}data} \PY{o}{=} \PY{n}{new\PYZus{}data}\PY{o}{.}\PY{n}{join}\PY{p}{(}\PY{n}{CDZ\PYZus{}hist}\PY{p}{[}\PY{p}{[}\PY{l+s+s1}{\PYZsq{}}\PY{l+s+s1}{Dividends}\PY{l+s+s1}{\PYZsq{}}\PY{p}{]}\PY{p}{]}\PY{p}{)}
\PY{n}{new\PYZus{}data}\PY{o}{.}\PY{n}{rename}\PY{p}{(}\PY{n}{columns}\PY{o}{=}\PY{p}{\PYZob{}}\PY{l+s+s1}{\PYZsq{}}\PY{l+s+s1}{Dividends}\PY{l+s+s1}{\PYZsq{}}\PY{p}{:} \PY{l+s+s1}{\PYZsq{}}\PY{l+s+s1}{CDZ.TO.DIV}\PY{l+s+s1}{\PYZsq{}}\PY{p}{\PYZcb{}}\PY{p}{,} \PY{n}{inplace}\PY{o}{=}\PY{k+kc}{True}\PY{p}{)}

\PY{n}{new\PYZus{}data} \PY{o}{=} \PY{n}{new\PYZus{}data}\PY{o}{.}\PY{n}{join}\PY{p}{(}\PY{n}{XIU\PYZus{}hist}\PY{p}{[}\PY{p}{[}\PY{l+s+s1}{\PYZsq{}}\PY{l+s+s1}{Dividends}\PY{l+s+s1}{\PYZsq{}}\PY{p}{]}\PY{p}{]}\PY{p}{)}
\PY{n}{new\PYZus{}data}\PY{o}{.}\PY{n}{rename}\PY{p}{(}\PY{n}{columns}\PY{o}{=}\PY{p}{\PYZob{}}\PY{l+s+s1}{\PYZsq{}}\PY{l+s+s1}{Dividends}\PY{l+s+s1}{\PYZsq{}}\PY{p}{:} \PY{l+s+s1}{\PYZsq{}}\PY{l+s+s1}{XIU.TO.DIV}\PY{l+s+s1}{\PYZsq{}}\PY{p}{\PYZcb{}}\PY{p}{,} \PY{n}{inplace}\PY{o}{=}\PY{k+kc}{True}\PY{p}{)}

\PY{n}{new\PYZus{}data} \PY{o}{=} \PY{n}{new\PYZus{}data}\PY{o}{.}\PY{n}{join}\PY{p}{(}\PY{n}{XDV\PYZus{}hist}\PY{p}{[}\PY{p}{[}\PY{l+s+s1}{\PYZsq{}}\PY{l+s+s1}{Dividends}\PY{l+s+s1}{\PYZsq{}}\PY{p}{]}\PY{p}{]}\PY{p}{)}
\PY{n}{new\PYZus{}data}\PY{o}{.}\PY{n}{rename}\PY{p}{(}\PY{n}{columns}\PY{o}{=}\PY{p}{\PYZob{}}\PY{l+s+s1}{\PYZsq{}}\PY{l+s+s1}{Dividends}\PY{l+s+s1}{\PYZsq{}}\PY{p}{:} \PY{l+s+s1}{\PYZsq{}}\PY{l+s+s1}{XDV.TO.DIV}\PY{l+s+s1}{\PYZsq{}}\PY{p}{\PYZcb{}}\PY{p}{,} \PY{n}{inplace}\PY{o}{=}\PY{k+kc}{True}\PY{p}{)}

\PY{n}{new\PYZus{}data} \PY{o}{=} \PY{n}{new\PYZus{}data}\PY{o}{.}\PY{n}{join}\PY{p}{(}\PY{n}{XTR\PYZus{}hist}\PY{p}{[}\PY{p}{[}\PY{l+s+s1}{\PYZsq{}}\PY{l+s+s1}{Dividends}\PY{l+s+s1}{\PYZsq{}}\PY{p}{]}\PY{p}{]}\PY{p}{)}
\PY{n}{new\PYZus{}data}\PY{o}{.}\PY{n}{rename}\PY{p}{(}\PY{n}{columns}\PY{o}{=}\PY{p}{\PYZob{}}\PY{l+s+s1}{\PYZsq{}}\PY{l+s+s1}{Dividends}\PY{l+s+s1}{\PYZsq{}}\PY{p}{:} \PY{l+s+s1}{\PYZsq{}}\PY{l+s+s1}{XTR.TO.DIV}\PY{l+s+s1}{\PYZsq{}}\PY{p}{\PYZcb{}}\PY{p}{,} \PY{n}{inplace}\PY{o}{=}\PY{k+kc}{True}\PY{p}{)}

\PY{n}{new\PYZus{}data} \PY{o}{=} \PY{n}{new\PYZus{}data}\PY{o}{.}\PY{n}{join}\PY{p}{(}\PY{n}{ZEB\PYZus{}hist}\PY{p}{[}\PY{p}{[}\PY{l+s+s1}{\PYZsq{}}\PY{l+s+s1}{Dividends}\PY{l+s+s1}{\PYZsq{}}\PY{p}{]}\PY{p}{]}\PY{p}{)}
\PY{n}{new\PYZus{}data}\PY{o}{.}\PY{n}{rename}\PY{p}{(}\PY{n}{columns}\PY{o}{=}\PY{p}{\PYZob{}}\PY{l+s+s1}{\PYZsq{}}\PY{l+s+s1}{Dividends}\PY{l+s+s1}{\PYZsq{}}\PY{p}{:} \PY{l+s+s1}{\PYZsq{}}\PY{l+s+s1}{ZEB.TO.DIV}\PY{l+s+s1}{\PYZsq{}}\PY{p}{\PYZcb{}}\PY{p}{,} \PY{n}{inplace}\PY{o}{=}\PY{k+kc}{True}\PY{p}{)}

\PY{n}{new\PYZus{}data} \PY{o}{=} \PY{n}{new\PYZus{}data}\PY{o}{.}\PY{n}{join}\PY{p}{(}\PY{n}{FIE\PYZus{}hist}\PY{p}{[}\PY{p}{[}\PY{l+s+s1}{\PYZsq{}}\PY{l+s+s1}{Dividends}\PY{l+s+s1}{\PYZsq{}}\PY{p}{]}\PY{p}{]}\PY{p}{)}
\PY{n}{new\PYZus{}data}\PY{o}{.}\PY{n}{rename}\PY{p}{(}\PY{n}{columns}\PY{o}{=}\PY{p}{\PYZob{}}\PY{l+s+s1}{\PYZsq{}}\PY{l+s+s1}{Dividends}\PY{l+s+s1}{\PYZsq{}}\PY{p}{:} \PY{l+s+s1}{\PYZsq{}}\PY{l+s+s1}{FIE.TO.DIV}\PY{l+s+s1}{\PYZsq{}}\PY{p}{\PYZcb{}}\PY{p}{,} \PY{n}{inplace}\PY{o}{=}\PY{k+kc}{True}\PY{p}{)}
\end{Verbatim}
\end{tcolorbox}

    \begin{tcolorbox}[breakable, size=fbox, boxrule=1pt, pad at break*=1mm,colback=cellbackground, colframe=cellborder]
\prompt{In}{incolor}{757}{\boxspacing}
\begin{Verbatim}[commandchars=\\\{\}]
\PY{n}{new\PYZus{}data}\PY{o}{.}\PY{n}{reset\PYZus{}index}\PY{p}{(}\PY{n}{level}\PY{o}{=}\PY{l+m+mi}{0}\PY{p}{,} \PY{n}{inplace}\PY{o}{=}\PY{k+kc}{True}\PY{p}{)}
\end{Verbatim}
\end{tcolorbox}

    \hypertarget{function-to-rollup-columns}{%
\subsubsection{Function to rollup
columns}\label{function-to-rollup-columns}}

    \begin{tcolorbox}[breakable, size=fbox, boxrule=1pt, pad at break*=1mm,colback=cellbackground, colframe=cellborder]
\prompt{In}{incolor}{758}{\boxspacing}
\begin{Verbatim}[commandchars=\\\{\}]
\PY{c+c1}{\PYZsh{} Function to add columns to dfReturns \PYZhy{} }
\PY{k}{def} \PY{n+nf}{Add\PYZus{}Columns}\PY{p}{(}\PY{n}{df}\PY{p}{,} \PY{n}{col\PYZus{}name}\PY{p}{)}\PY{p}{:}
    \PY{n}{min\PYZus{}indices} \PY{o}{=} \PY{n}{df}\PY{o}{.}\PY{n}{groupby}\PY{p}{(}\PY{l+s+s1}{\PYZsq{}}\PY{l+s+s1}{trade\PYZus{}year}\PY{l+s+s1}{\PYZsq{}}\PY{p}{)}\PY{p}{[}\PY{l+s+s1}{\PYZsq{}}\PY{l+s+s1}{Date}\PY{l+s+s1}{\PYZsq{}}\PY{p}{]}\PY{o}{.}\PY{n}{idxmin}\PY{p}{(}\PY{p}{)}
    \PY{n}{dfReturns}\PY{p}{[}\PY{n}{col\PYZus{}name}\PY{o}{+}\PY{l+s+s2}{\PYZdq{}}\PY{l+s+s2}{\PYZus{}start}\PY{l+s+s2}{\PYZdq{}}\PY{p}{]} \PY{o}{=} \PY{n+nb}{list}\PY{p}{(}\PY{n}{df}\PY{o}{.}\PY{n}{loc}\PY{p}{[}\PY{n}{min\PYZus{}indices}\PY{p}{]}\PY{p}{[}\PY{n}{col\PYZus{}name}\PY{p}{]}\PY{p}{)}
    \PY{n}{max\PYZus{}indices} \PY{o}{=} \PY{n}{df}\PY{o}{.}\PY{n}{groupby}\PY{p}{(}\PY{l+s+s1}{\PYZsq{}}\PY{l+s+s1}{trade\PYZus{}year}\PY{l+s+s1}{\PYZsq{}}\PY{p}{)}\PY{p}{[}\PY{l+s+s1}{\PYZsq{}}\PY{l+s+s1}{Date}\PY{l+s+s1}{\PYZsq{}}\PY{p}{]}\PY{o}{.}\PY{n}{idxmax}\PY{p}{(}\PY{p}{)}
    \PY{n}{dfReturns}\PY{p}{[}\PY{n}{col\PYZus{}name}\PY{o}{+}\PY{l+s+s2}{\PYZdq{}}\PY{l+s+s2}{\PYZus{}end}\PY{l+s+s2}{\PYZdq{}}\PY{p}{]} \PY{o}{=} \PY{n+nb}{list}\PY{p}{(}\PY{n}{df}\PY{o}{.}\PY{n}{loc}\PY{p}{[}\PY{n}{max\PYZus{}indices}\PY{p}{]}\PY{p}{[}\PY{n}{col\PYZus{}name}\PY{p}{]}\PY{p}{)}
    \PY{n}{dfReturns}\PY{p}{[}\PY{n}{col\PYZus{}name}\PY{o}{+}\PY{l+s+s2}{\PYZdq{}}\PY{l+s+s2}{\PYZus{}div}\PY{l+s+s2}{\PYZdq{}}\PY{p}{]} \PY{o}{=} \PY{n+nb}{list}\PY{p}{(}\PY{n}{df}\PY{p}{[}\PY{p}{[}\PY{l+s+s2}{\PYZdq{}}\PY{l+s+s2}{trade\PYZus{}year}\PY{l+s+s2}{\PYZdq{}}\PY{p}{,}\PY{n}{col\PYZus{}name}\PY{o}{+}\PY{l+s+s2}{\PYZdq{}}\PY{l+s+s2}{.DIV}\PY{l+s+s2}{\PYZdq{}}\PY{p}{]}\PY{p}{]}\PY{o}{.}\PYZbs{}
                                      \PY{n}{groupby}\PY{p}{(}\PY{l+s+s2}{\PYZdq{}}\PY{l+s+s2}{trade\PYZus{}year}\PY{l+s+s2}{\PYZdq{}}\PY{p}{)}\PY{o}{.}\PY{n}{sum}\PY{p}{(}\PY{p}{)}\PY{p}{[}\PY{n}{col\PYZus{}name}\PY{o}{+}\PY{l+s+s2}{\PYZdq{}}\PY{l+s+s2}{.DIV}\PY{l+s+s2}{\PYZdq{}}\PY{p}{]}\PY{p}{)}
\end{Verbatim}
\end{tcolorbox}

    \begin{tcolorbox}[breakable, size=fbox, boxrule=1pt, pad at break*=1mm,colback=cellbackground, colframe=cellborder]
\prompt{In}{incolor}{759}{\boxspacing}
\begin{Verbatim}[commandchars=\\\{\}]
\PY{n}{Add\PYZus{}Columns}\PY{p}{(}\PY{n}{new\PYZus{}data}\PY{p}{,}\PY{l+s+s2}{\PYZdq{}}\PY{l+s+s2}{CDZ.TO}\PY{l+s+s2}{\PYZdq{}}\PY{p}{)}
\PY{n}{Add\PYZus{}Columns}\PY{p}{(}\PY{n}{new\PYZus{}data}\PY{p}{,}\PY{l+s+s2}{\PYZdq{}}\PY{l+s+s2}{XIU.TO}\PY{l+s+s2}{\PYZdq{}}\PY{p}{)}
\PY{n}{Add\PYZus{}Columns}\PY{p}{(}\PY{n}{new\PYZus{}data}\PY{p}{,}\PY{l+s+s2}{\PYZdq{}}\PY{l+s+s2}{XDV.TO}\PY{l+s+s2}{\PYZdq{}}\PY{p}{)}
\PY{n}{Add\PYZus{}Columns}\PY{p}{(}\PY{n}{new\PYZus{}data}\PY{p}{,}\PY{l+s+s2}{\PYZdq{}}\PY{l+s+s2}{XTR.TO}\PY{l+s+s2}{\PYZdq{}}\PY{p}{)}
\PY{n}{Add\PYZus{}Columns}\PY{p}{(}\PY{n}{new\PYZus{}data}\PY{p}{,}\PY{l+s+s2}{\PYZdq{}}\PY{l+s+s2}{ZEB.TO}\PY{l+s+s2}{\PYZdq{}}\PY{p}{)}
\PY{n}{Add\PYZus{}Columns}\PY{p}{(}\PY{n}{new\PYZus{}data}\PY{p}{,}\PY{l+s+s2}{\PYZdq{}}\PY{l+s+s2}{FIE.TO}\PY{l+s+s2}{\PYZdq{}}\PY{p}{)}
\end{Verbatim}
\end{tcolorbox}

    \begin{tcolorbox}[breakable, size=fbox, boxrule=1pt, pad at break*=1mm,colback=cellbackground, colframe=cellborder]
\prompt{In}{incolor}{810}{\boxspacing}
\begin{Verbatim}[commandchars=\\\{\}]
\PY{c+c1}{\PYZsh{}dfReturns.iloc[0:3,0:4]}
\end{Verbatim}
\end{tcolorbox}

    \hypertarget{table-for-analysis}{%
\subsubsection{Table for Analysis}\label{table-for-analysis}}

    The final result is a table that shows the start and end closing prices,
and total dividends paid for each stock for each year.

Here are the first couple rows for the CDZ columns:

\begin{tabular}{rrrr}
\toprule
 Year &  CDZ.TO\_start &  CDZ.TO\_end &  CDZ.TO\_div \\
\midrule
 2011 &     20.760000 &   21.309999 &       0.688 \\
 2012 &     21.490000 &   22.440001 &       0.724 \\
 2013 &     22.690001 &   24.580000 &       0.855 \\
\bottomrule
\end{tabular}

    \hypertarget{calculate-net-worth-after-income-draw}{%
\subsubsection{Calculate Net Worth after Income
Draw}\label{calculate-net-worth-after-income-draw}}

    Now we are ready to ready to test each scenario.

\textbf{Here are the rules:}

\begin{itemize}
\item
  At the beginning of the first year buy \$1,000,000 of stock.
\item
  Draw \$40,000 in income from each stock at the end of each
  year.\(\footnote{We will consider this year to have ended today}\)
\item
  If the dividends pay more than \$40,000 reinvest the remainder, if
  they pay less, sell shares to make up the difference.
\item
  Let's see what is remaining for each stock.
\end{itemize}

The following function will do these calculations.

    \begin{tcolorbox}[breakable, size=fbox, boxrule=1pt, pad at break*=1mm,colback=cellbackground, colframe=cellborder]
\prompt{In}{incolor}{805}{\boxspacing}
\begin{Verbatim}[commandchars=\\\{\}]
\PY{k}{def} \PY{n+nf}{calcRemain}\PY{p}{(}\PY{n}{stock\PYZus{}name}\PY{p}{)}\PY{p}{:}
    \PY{c+c1}{\PYZsh{} variables}
    \PY{n}{initial\PYZus{}amt} \PY{o}{=} \PY{l+m+mi}{1\PYZus{}000\PYZus{}000}
    \PY{n}{year\PYZus{}income} \PY{o}{=} \PY{l+m+mi}{40\PYZus{}000}
    
    \PY{n}{result} \PY{o}{=} \PY{p}{[}\PY{p}{(}\PY{n}{y}\PY{p}{,}\PY{n}{s}\PY{p}{,}\PY{n}{e}\PY{p}{,}\PY{n}{d}\PY{p}{)} \PY{k}{for} \PY{n}{y}\PY{p}{,}\PY{n}{s}\PY{p}{,}\PY{n}{e}\PY{p}{,}\PY{n}{d} \PY{o+ow}{in} \PY{n+nb}{zip}\PY{p}{(}\PY{n}{dfReturns}\PY{p}{[}\PY{l+s+s1}{\PYZsq{}}\PY{l+s+s1}{Year}\PY{l+s+s1}{\PYZsq{}}\PY{p}{]}\PY{p}{,} \PY{n}{dfReturns}\PY{p}{[}\PY{n}{stock\PYZus{}name}\PY{o}{+}\PY{l+s+s1}{\PYZsq{}}\PY{l+s+s1}{\PYZus{}start}\PY{l+s+s1}{\PYZsq{}}\PY{p}{]}\PY{p}{,} \PYZbs{}
                                     \PY{n}{dfReturns}\PY{p}{[}\PY{n}{stock\PYZus{}name}\PY{o}{+}\PY{l+s+s1}{\PYZsq{}}\PY{l+s+s1}{\PYZus{}end}\PY{l+s+s1}{\PYZsq{}}\PY{p}{]}\PY{p}{,}\PY{n}{dfReturns}\PY{p}{[}\PY{n}{stock\PYZus{}name}\PY{o}{+}\PY{l+s+s1}{\PYZsq{}}\PY{l+s+s1}{\PYZus{}div}\PY{l+s+s1}{\PYZsq{}}\PY{p}{]}\PY{p}{)}\PY{p}{]}
    
    \PY{n}{initial\PYZus{}price} \PY{o}{=} \PY{n}{result}\PY{p}{[}\PY{l+m+mi}{0}\PY{p}{]}\PY{p}{[}\PY{l+m+mi}{1}\PY{p}{]}
    \PY{n}{no\PYZus{}shares} \PY{o}{=} \PY{n}{mt}\PY{o}{.}\PY{n}{trunc}\PY{p}{(}\PY{n}{initial\PYZus{}amt}\PY{o}{/}\PY{n}{initial\PYZus{}price}\PY{p}{)}

    \PY{k}{for} \PY{n}{row} \PY{o+ow}{in} \PY{n}{result}\PY{p}{:}
        \PY{n}{div\PYZus{}income} \PY{o}{=} \PY{n}{no\PYZus{}shares} \PY{o}{*} \PY{n}{row}\PY{p}{[}\PY{l+m+mi}{3}\PY{p}{]}
        
        \PY{k}{if} \PY{n}{div\PYZus{}income} \PY{o}{\PYZgt{}} \PY{n}{year\PYZus{}income}\PY{p}{:}
            \PY{n}{end\PYZus{}price} \PY{o}{=} \PY{n}{row}\PY{p}{[}\PY{l+m+mi}{2}\PY{p}{]}
            \PY{n}{buy\PYZus{}shares} \PY{o}{=} \PY{n}{mt}\PY{o}{.}\PY{n}{trunc}\PY{p}{(}\PY{p}{(}\PY{n}{div\PYZus{}income} \PY{o}{\PYZhy{}} \PY{n}{year\PYZus{}income}\PY{p}{)}\PY{o}{/}\PY{n}{end\PYZus{}price}\PY{p}{)}
            \PY{n}{no\PYZus{}shares} \PY{o}{=} \PY{n}{no\PYZus{}shares} \PY{o}{+} \PY{n}{buy\PYZus{}shares}
            
        \PY{k}{else}\PY{p}{:}
            \PY{n}{end\PYZus{}price} \PY{o}{=} \PY{n}{row}\PY{p}{[}\PY{l+m+mi}{2}\PY{p}{]}
            \PY{n}{sell\PYZus{}shares} \PY{o}{=} \PY{n}{mt}\PY{o}{.}\PY{n}{trunc}\PY{p}{(}\PY{p}{(}\PY{n}{year\PYZus{}income} \PY{o}{\PYZhy{}} \PY{n}{div\PYZus{}income}\PY{p}{)}\PY{o}{/}\PY{n}{end\PYZus{}price}\PY{p}{)}
            \PY{n}{no\PYZus{}shares} \PY{o}{=} \PY{n}{no\PYZus{}shares} \PY{o}{\PYZhy{}} \PY{n}{sell\PYZus{}shares}
            
    \PY{n}{final\PYZus{}amt} \PY{o}{=} \PY{n}{result}\PY{p}{[}\PY{o}{\PYZhy{}}\PY{l+m+mi}{1}\PY{p}{]}\PY{p}{[}\PY{l+m+mi}{2}\PY{p}{]} \PY{o}{*} \PY{n}{no\PYZus{}shares}
    \PY{k}{return}\PY{p}{(}\PY{n}{final\PYZus{}amt}\PY{p}{)}
\end{Verbatim}
\end{tcolorbox}

    Let's apply the function to each of the stocks.

    \begin{tcolorbox}[breakable, size=fbox, boxrule=1pt, pad at break*=1mm,colback=cellbackground, colframe=cellborder]
\prompt{In}{incolor}{806}{\boxspacing}
\begin{Verbatim}[commandchars=\\\{\}]
\PY{n}{stock\PYZus{}list} \PY{o}{=} \PY{p}{[}\PY{p}{]}
\PY{n}{stock\PYZus{}list}\PY{o}{.}\PY{n}{append}\PY{p}{(}\PY{n}{mt}\PY{o}{.}\PY{n}{trunc}\PY{p}{(}\PY{n}{calcRemain}\PY{p}{(}\PY{l+s+s1}{\PYZsq{}}\PY{l+s+s1}{CDZ.TO}\PY{l+s+s1}{\PYZsq{}}\PY{p}{)}\PY{p}{)}\PY{p}{)}
\PY{n}{stock\PYZus{}list}\PY{o}{.}\PY{n}{append}\PY{p}{(}\PY{n}{mt}\PY{o}{.}\PY{n}{trunc}\PY{p}{(}\PY{n}{calcRemain}\PY{p}{(}\PY{l+s+s1}{\PYZsq{}}\PY{l+s+s1}{XIU.TO}\PY{l+s+s1}{\PYZsq{}}\PY{p}{)}\PY{p}{)}\PY{p}{)}
\PY{n}{stock\PYZus{}list}\PY{o}{.}\PY{n}{append}\PY{p}{(}\PY{n}{mt}\PY{o}{.}\PY{n}{trunc}\PY{p}{(}\PY{n}{calcRemain}\PY{p}{(}\PY{l+s+s1}{\PYZsq{}}\PY{l+s+s1}{XDV.TO}\PY{l+s+s1}{\PYZsq{}}\PY{p}{)}\PY{p}{)}\PY{p}{)}
\PY{n}{stock\PYZus{}list}\PY{o}{.}\PY{n}{append}\PY{p}{(}\PY{n}{mt}\PY{o}{.}\PY{n}{trunc}\PY{p}{(}\PY{n}{calcRemain}\PY{p}{(}\PY{l+s+s1}{\PYZsq{}}\PY{l+s+s1}{XTR.TO}\PY{l+s+s1}{\PYZsq{}}\PY{p}{)}\PY{p}{)}\PY{p}{)}
\PY{n}{stock\PYZus{}list}\PY{o}{.}\PY{n}{append}\PY{p}{(}\PY{n}{mt}\PY{o}{.}\PY{n}{trunc}\PY{p}{(}\PY{n}{calcRemain}\PY{p}{(}\PY{l+s+s1}{\PYZsq{}}\PY{l+s+s1}{ZEB.TO}\PY{l+s+s1}{\PYZsq{}}\PY{p}{)}\PY{p}{)}\PY{p}{)}
\PY{n}{stock\PYZus{}list}\PY{o}{.}\PY{n}{append}\PY{p}{(}\PY{n}{mt}\PY{o}{.}\PY{n}{trunc}\PY{p}{(}\PY{n}{calcRemain}\PY{p}{(}\PY{l+s+s1}{\PYZsq{}}\PY{l+s+s1}{FIE.TO}\PY{l+s+s1}{\PYZsq{}}\PY{p}{)}\PY{p}{)}\PY{p}{)}
\end{Verbatim}
\end{tcolorbox}

    \begin{tcolorbox}[breakable, size=fbox, boxrule=1pt, pad at break*=1mm,colback=cellbackground, colframe=cellborder]
\prompt{In}{incolor}{807}{\boxspacing}
\begin{Verbatim}[commandchars=\\\{\}]
\PY{n}{dfRemains} \PY{o}{=} \PY{n}{pd}\PY{o}{.}\PY{n}{DataFrame}\PY{p}{(}\PY{n}{stock\PYZus{}list}\PY{p}{,} \PY{n}{index} \PY{o}{=}\PY{p}{[}\PY{l+s+s1}{\PYZsq{}}\PY{l+s+s1}{CDZ}\PY{l+s+s1}{\PYZsq{}}\PY{p}{,} \PY{l+s+s1}{\PYZsq{}}\PY{l+s+s1}{XIU}\PY{l+s+s1}{\PYZsq{}}\PY{p}{,} \PY{l+s+s1}{\PYZsq{}}\PY{l+s+s1}{XDV}\PY{l+s+s1}{\PYZsq{}}\PY{p}{,} \PY{l+s+s1}{\PYZsq{}}\PY{l+s+s1}{XTR}\PY{l+s+s1}{\PYZsq{}}\PY{p}{,}\PY{l+s+s1}{\PYZsq{}}\PY{l+s+s1}{ZEB}\PY{l+s+s1}{\PYZsq{}}\PY{p}{,}\PY{l+s+s1}{\PYZsq{}}\PY{l+s+s1}{FIE}\PY{l+s+s1}{\PYZsq{}}\PY{p}{]}\PY{p}{,} 
                                              \PY{n}{columns} \PY{o}{=}\PY{p}{[}\PY{l+s+s1}{\PYZsq{}}\PY{l+s+s1}{Remaining}\PY{l+s+s1}{\PYZsq{}}\PY{p}{]}\PY{p}{)} 
\end{Verbatim}
\end{tcolorbox}

    \begin{tcolorbox}[breakable, size=fbox, boxrule=1pt, pad at break*=1mm,colback=cellbackground, colframe=cellborder]
\prompt{In}{incolor}{808}{\boxspacing}
\begin{Verbatim}[commandchars=\\\{\}]
\PY{n}{dfRemains}\PY{o}{.}\PY{n}{sort\PYZus{}values}\PY{p}{(}\PY{n}{by}\PY{o}{=}\PY{p}{[}\PY{l+s+s1}{\PYZsq{}}\PY{l+s+s1}{Remaining}\PY{l+s+s1}{\PYZsq{}}\PY{p}{]}\PY{p}{,} \PY{n}{ascending}\PY{o}{=}\PY{k+kc}{False}\PY{p}{)}\PY{o}{.}\PY{n}{style}\PY{o}{.}\PY{n}{format}\PY{p}{(}\PY{l+s+s1}{\PYZsq{}}\PY{l+s+s1}{\PYZdl{}}\PY{l+s+si}{\PYZob{}0:,.2f\PYZcb{}}\PY{l+s+s1}{\PYZsq{}}\PY{p}{)}
\end{Verbatim}
\end{tcolorbox}

            \begin{tcolorbox}[breakable, size=fbox, boxrule=.5pt, pad at break*=1mm, opacityfill=0]
\prompt{Out}{outcolor}{808}{\boxspacing}
\begin{Verbatim}[commandchars=\\\{\}]
<pandas.io.formats.style.Styler at 0x7fca41a03970>
\end{Verbatim}
\end{tcolorbox}
        
    \hypertarget{and-the-winner-is}{%
\subsubsection{And The Winner is:}\label{and-the-winner-is}}

    The following table shows the remaining funds had we invested
\$1,000,000 in each of these stocks and withdrew \$40,000 in income each
year.

\begin{tabular}{lr}
\toprule
ETF &  Remaining \\
\midrule
ZEB &    \$2,071,776 \\
FIE &    \$1,521,226 \\
CDZ &    \$1,432,213 \\
XTR &    \$1,413,755 \\
XDV &    \$1,400,908 \\
XIU &    \$1,289,416 \\
\bottomrule
\end{tabular}

     There's a large difference in each of the ETFs. ZEB, BMO's Equal
Weighted Bank Stock ETF is the clear winner. If we had invested
\$1,000,000 ten years ago in ZEB and withdrew \$40,000 each year we
would now have over \$2,000,000 in the fund. Remarkable. It seems that
Banks ruled, at least for the last decade.

The remaining funds are relatively close. The difference between the
growth of FIE, the second ranked fund, and XIU the worst fund, is only
\$230,000. With every fund examined it was possible to take \$40,000 in
income each year and still keep all of your principle.

    \hypertarget{conclusions}{%
\section{Conclusions}\label{conclusions}}

    It seems it was the decade of Canadian bank stocks. We were able to draw
a healthy income from ZEB and also more than double our investment. It's
important to note that the last decade's results may not continue into
the next ten years. We faced a time when interest rates have steadily
fallen, Canadian stocks have underperformed their American counterparts,
and growth of the tech sector has been phenomenal. It's likely that the
macro conditions that brought about these results won't continue into
future decades.

However Canadian banks have been incredibly robust over many years. It's
amazing to think that BMO, Scotiabank, Toronto Dominion, CIBC and The
Royal Bank have been consistently paying dividends for over 100 years!
In fact the Bank of Nova Scotia and The Bank of Montreal have almost
been paying for 200 years. Our banks have a history of stability,
regulatory prudence, and consistency that is admirable.

Its also important to remember that there are tax considerations for the
types of income one receives from stock. Income from dividend stocks
such as CDZ, XDV and ZEB are eligible for a dividend tax credit. Over
90\% of the distributions from these ETFs are \textbf{eligible
dividends}. That's not true for many income funds - for XTR it's only
about 25\%. Consider that you can make over \$50,000 in income each year
from dividends and you won't pay a cent in tax, provided it's your only
source of
income.\(\footnote{Financial Post, April 11th, 2017, "You can earn \$50K in tax-free dividends, but there's a catch: You can't have a job"}\)

We cannot know the future, but it would probably be a good bet to invest
in Canadian bank stocks if one wanted consistent monthly income with
long term growth of capital.


    % Add a bibliography block to the postdoc
    
    
    
\end{document}
